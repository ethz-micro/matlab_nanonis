%!TEX root = matlabDoc.tex
\section{Installation and Use}

\subsection{Installation}

The preferred way to access the NanoLib is to make use of the free version control system Git.

\subsubsection{Git} \label{git-install}
If you are familiar to git, you can directly clone the repository to your workstation. With the command: \\
\texttt{git clone \url{https://github.com/ethz-micro/matlab_nanonis} ~/mynanolib} you will clone the repository to a subdirectory called \texttt{mynanonlib} in your actual working directory.
\subsubsection{Manual Download} \label{man-install}
If you do not know git probably the best is to simply download the ZIP file from the homepage. You will find the file here: \url{https://github.com/ethz-micro/matlab_nanonis} and click on the green field called ``Clone or download'', then ``Download ZIP''.

You can clone the repository to your computer or download the ZIP file and unzip it to a directory of your choice, for example: ```~/myFunctions```.

%In order to use and access the library you need to add the path to the directory where you copied the library with the ```addpath``` standard function of MatLab. Before you begin with your project just type in the MatLab prompt:
%```
%>>> addpath ~/myFunctions
%```
%
%You may use the functions of the library by calling first the ''class'' and then the specific function. For example, in order to load a SXM image, you can simply write ```load.loadSXM```.
%
%A simple example of how to load and to plot an image is given below:
%
%```matlab
%% load file
%fileName = 'SXM_file.sxm';
%sxmFile = loadSxM.loadProcessedSxM(fileName);
%
%%% plot data
%iCh = 1; % Channel number
%
%%plot image
%figure('Name',sprintf('file: %s',fileName));
%plotSxM.plotFile(sxmFile,iCh);
%```
%
%This and other examples can be found in the section *Example*.
